\documentclass[a4paper,10pt]{article}
\usepackage[utf8]{inputenc}
\usepackage{graphicx}
\usepackage{geometry}
\usepackage{hyperref}

% Configurar el diseño de la página
\geometry{margin=1in}

% Color para los enlaces
\usepackage{xcolor}
\hypersetup{
	colorlinks=true,
	linkcolor=blue,
	urlcolor=cyan
}

\begin{document}
	
	% Encabezado
	\begin{center}
		{\LARGE \textbf{Jabel Resendiz Aguirre}} \\
		\vspace{5pt}
		\textit{Estudiante de Ciencias de la Computación, Universidad de La Habana} \\
		\vspace{5pt}
		\href{https://t.me/resendizjr}{\includegraphics[width=10pt]{telegram.png} Telegram} \hspace{15pt}
		\href{mailto:jabelresendiz03@gmail.com}{\includegraphics[width=10pt]{gmail.png} Correo} \hspace{15pt}
		\href{https://www.linkedin.com/in/jabel-resendiz-aguirre-26123a261}{\includegraphics[width=10pt]{linkedin.png} LinkedIn}\hspace{15pt}
		\href{https://github.com/JabelResendiz}{\includegraphics[width=10pt]{github-mark.png} Github} \\
		\vspace{10pt}
		\textbf{Ubicación:} México (Disponibilidad para trabajo remoto o presencial) 
		\rule{\textwidth}{2pt}\vspace{5pt}
	\end{center}
	
	% Sección: Objetivo
	\section*{Resumen Profesional}
	Estudiante de 4to año de Ciencias de la Computación de la Facultad de Matema\'aticas y Computaci\'on con gran interés y conocimientos en desarrollo backend (C\#/.NET, Node.js), análisis de datos (Python, R) e Inteligencia Artificial (IA, NLP, RAG). Demostrada habilidad algorítmica y de resolución de problemas como Finalista del ICPC Caribe y medallista en Olimpiadas Internacionales de Matemáticas. Busco oportunidades desafiantes para aplicar mis habilidades en la creación de software innovador .
	
	% Sección: Habilidades y Conocimientos
\section*{Habilidades Técnicas}
\begin{itemize}
	\item \textbf{Lenguajes de Programación:}
	\begin{itemize}
		\item Python, C, C++, C\#, Haskell, Prolog, JavaScript, TypeScript
		\item \textbf{Go} (en aprendizaje activo), \textbf{R} (análisis estadístico).
	\end{itemize}
	
	
	\item \textbf{Desarrollo Backend:}
	\begin{itemize}
		\item \textbf{.NET} (C\#, incluyendo uso de ORMs como Entity Framework, y Blazor Server para UI renderizada en servidor).
		\item Node.js (Express.js).
		\item Diseño y desarrollo de APIs RESTful (\textbf{Clean Architecture}).
		\item \textbf{Django} (en aprendizaje activo).
	\end{itemize}
	
	\item \textbf{Bases de Datos:}
	\begin{itemize}
		\item \textbf{MySQL , PostgreSQL}(Relacionales).
		\item GraphQL (APIs de Datos)
	\end{itemize}
	
	\item \textbf{Desarrollo Frontend:}
	\begin{itemize}
		\item HTML, CSS, JavaScript (Vue.js), TypeScript.
		\item \textbf{Blazor WebAssembly} (componente UI interactivos en el navegador)
	\end{itemize}
	
	\item \textbf{Inteligencia Artificial y Análisis de Datos Avanzado:}
	\begin{itemize}
		\item \textbf{Optimización Matemática:} Algoritmos Simplex, Dual Simplex, Programación Lineal, Metaheurísticas (Descenso por Gradiente, Recocido Simulado, Algoritmos Evolutivos).
		\item \textbf{Procesamiento de Lenguaje Natural (NLP):} Retrieval Augmented Generation (RAG), Modelos de Lenguaje (LLMs).
		\item \textbf{Ingeniería de Datos:} Análisis Exploratorio de Datos (EDA), Web Scraping (extracción y ETL).
	\end{itemize}
	
	\item \textbf{Arquitecturas y Conceptos Fundamentales:}
	\begin{itemize}
		\item \textbf{Metodologías Ágiles (Scrum, Kanban).}
		\item Clean Architecture.
		\item Diseño de Algoritmos y Complejidad Computacional (incluyendo algoritmos aleatorizados).
		\item Diseño y Desarrollo de Compiladores (experiencia en Frontend, LLVM).
		\item Redes de Computadoras (particularmente el estudio del modelo TCP/IP),Arquitectura de Computadoras.
		\item Probabilidades y Estad\'isticas (conocimientos fundamentales - avanzados)
	\end{itemize}
	
	\item \textbf{Herramientas y Entornos de Desarrollo (DevOps):}
	\begin{itemize}
		\item Git, GitHub, Docker, Visual Studio Code, VirtualBox, Shell.
		\item LaTeX (documentación técnica y académica).
		\item Sistemas Operativos: Windows, Linux (Ubuntu).
	\end{itemize}
	
	\item \textbf{Habilidades Blandas:}
	\begin{itemize}
		\item Resolución de Problemas.
		\item Trabajo en Equipo y Colaboración.
		\item Pensamiento Analítico.
		\item Adaptabilidad y Aprendizaje Continuo.
	\end{itemize}
\end{itemize}
	
	% Sección: Proyectos Académicos
	\section*{Proyectos Académicos}
	\begin{itemize}
		\item \textbf{API de Gestión de Bajas Técnicas} (2024)
		Diseñé e implementé una \textbf{API RESTful escalable} utilizando \textbf{.NET (C\#)} y \textbf{MySQL} como sistema de gesti\'on de bases de datos, aplicando principios de \textbf{Clean Architecture}. Esta solución optimizó el proceso de gestión de inventario empresarial, mejorando la eficiencia y la mantenibilidad del sistema.
		
		\item \textbf{Plataforma Web para Gestión Empresarial} (2024)
		Desarrollé la interfaz de usuario de una \textbf{plataforma web intuitiva}(Vue.js), la cual se integró fluidamente con la API de Bajas Técnicas. Este proyecto mejoró significativamente la experiencia del usuario para la gestión operativa y el funcionamiento de la empresa.

		\item \textbf{Análisis Predictivo en Fórmula 1} (2024)
		Realicé un proyecto de \textbf{análisis exploratorio de datos (EDA)} y \textbf{modelado predictivo} sobre datos de Fórmula 1. Utilicé \textbf{Python} con \textbf{pandas} y \textbf{matplotlib} para identificar patrones de rendimiento y tendencias.
		
		
	\item \textbf{Intérprete y Compilador del Lenguaje HULK} (2023)
	Inicialmente, construí un intérprete de consola para el lenguaje de programación HULK en \textbf{.NET (C\#)}, el cual posteriormente extendí a una \textbf{versión web} para una interacción más dinámica. En una fase avanzada, desarrollé un \textbf{compilador} completo para HULK, implementando el \textbf{frontend en C} con herramientas como \textbf{Flex (.l)} y \textbf{Bison (.y)} para el análisis léxico y sintáctico, y generando el \textbf{backend con LLVM} para la optimización y generación de código de máquina. 
		
		
		
		\item \textbf{Intérprete de Lenguaje Funcional Avanzado} (Haskell)
		Desarrollé un intérprete para un lenguaje puramente funcional en \textbf{Haskell}, diseñado específicamente para \textbf{cálculos matemáticos complejos}. Este proyecto profundizó mi comprensión en paradigmas de programación funcional y procesamiento de lenguajes.
		
		
		\item \textbf{Implementación de Cliente y Servidor HTTPS} (Redes de Computadoras)
		Como parte del curso de Redes de Computadoras, implementé desde cero un \textbf{cliente y servidor HTTPS b\'asicos}, aplicando conocimientos fundamentales de protocolos de red, cifrado y seguridad web.
		
		
		  \item \textbf{Asistente Turístico Inteligente para Museos con IA y Optimización} (En curso, Finales de 2025)
		 Liderando el desarrollo de un innovador asistente turístico especializado en museos, este proyecto integra \textbf{Inteligencia Artificial y técnicas de simulación} para optimizar la experiencia del usuario. El sistema utiliza \textbf{Web Scraping} para la recopilación dinámica de información relevante sobre museos. La creación de itinerarios turísticos personalizados se potencia mediante la aplicación de \textbf{algoritmos genéticos}. Se incorpora \textbf{Inteligencia Artificial Generativa (IAG)} a través de la integración de \textbf{LLMs como MistralAI} para generar respuestas y contenido enriquecido. Además, se implementa \textbf{búsqueda semántica} utilizando \textbf{embeddings vectoriales (Sentence Transformers)} para mejorar la precisión de las consultas. La funcionalidad se extiende con la integración de \textbf{APIs de geolocalización y clima} para ofrecer widgets dinámicos y contextuales. La persistencia y consulta de datos se gestiona eficientemente con una \textbf{base de datos GraphQL}.
		 
		
		
	\end{itemize}
	
	\section*{Experiencia Extracurricular}
	\begin{itemize}
		\item \textbf{Desarrollador de Software en Prácticas} | CITMATEL | La Habana, Cuba
		[Abril] 2024 – [Junio] 2025
		\begin{itemize}
			\item Contribuí activamente en la adaptación y modificación de historias de usuario para una nueva versión de una aplicación empresarial crítica, enfocada en módulos de nóminas.
			\item Participé en la detección y corrección de errores (debugging y refactoring de código) para asegurar la calidad y estabilidad de la aplicación, facilitando su preparación para el lanzamiento de la nueva versión.
			\item Adquirí experiencia práctica en el ciclo de vida del desarrollo de software y metodologías ágiles en un entorno empresarial.
		\end{itemize}
		
		\item \textbf{Ingeniero de Software (Motor Antivirus) en Prácticas} | SEGURMÁTICA | La Habana, Cuba
		Marzo 2024 – Octubre 2024
		\begin{itemize}
			\item Realicé \textbf{ingeniería inversa} de binarios y \textbf{análisis de malware} para comprender técnicas de ofuscación y funcionalidades de amenazas de seguridad.
			\item Contribuí directamente al departamento de desarrollo del motor antivirus, aplicando los conocimientos obtenidos para mejorar las capacidades de detección y protección del producto.
			\item Gana experiencia valiosa en seguridad informática, análisis de sistemas y programación de bajo nivel en un entorno de ciberseguridad.
		\end{itemize}
	\end{itemize}
	
	% Sección: Educación
	\section*{Educación}
	\begin{itemize}
		\item \textbf{Estudiante de Ciencias de la Computación} | Universidad de La Habana | La Habana, Cuba
		Abril 2022 – Julio 2026 (Fecha de Graduación Prevista)
		\item \textbf{GPA} : 4.89/5 (Hasta finalizado el 6to semestre)
	\end{itemize}
	
	\section*{Premios y Reconocimientos}
	\begin{itemize}
		\item \textbf{Finalista del ICPC Caribe} | International Collegiate Programming Contest (ICPC) | 2024
		Destacada participación en la competición de programación universitaria más prestigiosa. Logramos el 3er lugar a nivel de Universidad y 5to lugar en el clasificatorio para la Final del ICPC Caribe.
		
		\item \textbf{Medallista en Olimpiadas Internacionales de Matemáticas}
		Miembro 3 veces de la Preselección Nacional de Matemáticas,con una medalla de plata en la Olimpiada Centroamericana de Matemáticas 2019, y dos medallas de bronce en la Olimpiada Iberoamericana de Matemáticas (2020, 2021).
	\end{itemize}
	
	% Sección: Idiomas
	\section*{Idiomas}
	\begin{itemize}
		\item \textbf{Español:} Nativo
		\item \textbf{Inglés:} Intermedio (B1)
	\end{itemize}
	
	
	
	
\end{document}
